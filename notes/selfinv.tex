\let\safetikz\shipout
\input pgf
\input tikz
\usetikzlibrary{calc}
\let\shipout\safetikz

\magnification\magstephalf
\parskip3pt
\baselineskip14pt

\font\ninerm=cmr9
\font\eightrm=cmr8
\font\sixrm=cmr6

\font\ninei=cmmi9  \skewchar\ninei='177
\font\eighti=cmmi8  \skewchar\eighti='177
\font\sixi=cmmi6  \skewchar\sixi='177

\font\tenbi=cmmib10  \skewchar\tenbi='177
\font\ninebi=cmmib9  \skewchar\ninebi='177

\font\ninesy=cmsy9  \skewchar\ninesy='60
\font\eightsy=cmsy8  \skewchar\eightsy='60
\font\sixsy=cmsy6  \skewchar\sixsy='60

\font\tenbsy=cmbsy10  \skewchar\tenbsy='60
\font\sevenbsy=cmbsy7  \skewchar\sevenbsy='60
\font\fivebsy=cmbsy5  \skewchar\fivebsy='60

\font\elevenex=cmex10 scaled\magstephalf
\font\nineex=cmex9
\font\eightex=cmex8
\font\sevenex=cmex7

\font\ninebf=cmbx9
\font\eightbf=cmbx8
\font\sixbf=cmbx6

\font\tenthinbf=cmb10
\font\ninethinbf=cmb10 at 9.25pt
\font\eightthinbf=cmb10 at 8.5pt

\font\twelvett=cmtt12  \hyphenchar\twelvett=-1  % inhibit hyphenation in tt
\font\tensltt=cmsltt10  \hyphenchar\tensltt=-1
\font\ninett=cmtt9  \hyphenchar\ninett=-1
\font\ninesltt=cmsltt10 at 9pt  \hyphenchar\ninesltt=-1
\font\eighttt=cmtt8  \hyphenchar\eighttt=-1
\font\seventt=cmtt8 scaled 875  \hyphenchar\seventt=-1

\font\ninesl=cmsl9
\font\eightsl=cmsl8

\font\nineit=cmti9
\font\eightit=cmti8
\font\sevenit=cmti8 scaled 900

\font\ninessbx=cmssbx9
\font\niness=cmssq9
\font\eightss=cmssq8
\font\eightssi=cmssqi8
\font\eightssbx=cmssbx8
\font\eightssi=cmssi8
\font\fourss=cmssq8 scaled 600
\font\sixss=cmssq8 scaled 800
\font\sevenss=cmssq8 scaled 900
\font\sevenssbx=cmssbx8 scaled 900
\font\sixssbx=cmssbx8 scaled 800
\font\tenss=cmss10
\font\tenssbx=cmssbx10
\font\twelvess=cmss12
\font\titlefont=cmssbx10 scaled\magstep2

\font\tencsc=cmcsc10

\font\manfnt=manfnt % special symbols from the TeX project

\def\ninepoint{\def\rm{\fam0\ninerm}%
  \textfont0=\ninerm \scriptfont0=\sixrm \scriptscriptfont0=\fiverm
  \textfont1=\ninei \scriptfont1=\sixi \scriptscriptfont1=\fivei
  \textfont2=\ninesy \scriptfont2=\sixsy \scriptscriptfont2=\fivesy
  \textfont3=\nineex \scriptfont3=\sevenex \scriptscriptfont3=\sevenex
  \def\it{\fam\itfam\nineit}%
  \textfont\itfam=\nineit
  \def\sl{\fam\slfam\let\ninett=\ninesltt\ninesl}%
  \textfont\slfam=\ninesl
  \def\bf{\fam\bffam\ninebf}%
  \textfont\bffam=\ninebf \scriptfont\bffam=\sixbf
   \scriptscriptfont\bffam=\fivebf
  \def\tt{\fam\ttfam\ninett}%
  \let\sltt=\error
  \textfont\ttfam=\ninett
  \def\oldstyle{\fam\@ne\ninei}%
  \normalbaselineskip=11pt\rm}

\def\slug{\hbox{\kern1.5pt\vrule width2.5pt height6pt depth1.5pt\kern1.5pt}}
\def\slugonright{\vrule width0pt\nobreak\hfill\slug}

\centerline{\bf Self-inverse functions and palindromic circuits}
\centerline{--- {\it Preliminary notes on our discussions\/} ---}
\centerline{Mathias Soeken}
\centerline{September 24 -- \dots, 2014}

\bigskip
\bigskip

\noindent{\bf 1. Self-inverse functions.}\enspace A reversible function is
called {\it self-inverse\/} if $f(f(x))=x$ for all input assignments $x$, or in
other words if $f=f^{-1}$.  It helps a lot to investigate the respective
permutations that are represented by the reversible functions, i.e.~elements
from the symmetric group $S_{2^n}$.  Then, self-inverse functions correspond to
self-conjugate permutations that are also often called {\it involutions\/} in
the literature.  The permutation that represents the identity is denoted
$\pi_{\rm id}$.  The permutation matrix of an involution is symmetric.  There is
an interesting property of involutions.

\smallskip \noindent{\bf Lemma 1.} \sl Let $f$ be a self-inverse function and
$\pi_f$ its corresponding permutation.  Then, the cycle representation of
$\pi_f$ consists only of transpositions or fixpoints. \rm

\smallskip\noindent {\it Proof.} The cycle representation is unique when
disregarding order of cycles and order of elements within cycles.  Assume that
the cycle representation of $\pi_f$ consists of a cycle $(i_1\; i_2\; \ldots \;
i_k)$ with $k>2$.  Then $\pi_f^{-1}$ consists of the cycle $(i_k \; \ldots \;
i_2 \; i_1)$ and hence $\pi_f\neq \pi_f^{-1}$. \qquad\slug

\smallskip\noindent We define some special classes of self-inverse functions:

\item{--} A self-inverse function $f$ is {\it trivial\/} if $f$ is the identity
function.
\item{--} A self-inverse function $f$ is {\it simple\/} if $f$ can be realized
using a single mixed-polarity multiple-controlled Toffoli gate.
\item{--} A self-inverse function $f$ is a {\it transposition\/} if its
permutation representation $\pi_f$ is a transposition.
\item{--} A self-inverse function $f$ is called {\it palindromic\/} if the
number of cycles in its permutation representation $\pi_f$ is $2^k$ for some
nonnegative integer $k$. 

\smallskip\noindent From the book of T.~Muir [{\sl A Treatise on the Theory of
Determinants\/} (1960)] we have that the number of involutions on $n$ elements
is
$$ I(n) = 1 + \sum_{k=0}^{\lceil(n-1)/2\rceil}{1\over{(k+1)!}}
          \prod_{i=0}^k\left(\matrix{ n-2i \cr 2}\right), \eqno(1) $$ or
alternatively according to S.~Skiena [{\sl Implementing Discrete Mathematics\/}
(1990)] we have
$$ I(n) = n!\sum_{k=0}^{\lceil n/2\rceil}{1\over{2^kk!(n-2k)!}}. \eqno(2) $$
The number of self-inverse functions on $n$ variables is $I(2^n)$.  From (2) it
can easily be seen that the percentage of self-inverse functions with respect to
all reversible functions is
$$ {I(2^n)\over{2^n!}} = \sum_{k=0}^{2^{n-1}}{1\over{2^kk!(2^n-2k)!}}. \eqno(3) $$
The number of simple self-inverse functions is $n\cdot3^{n-1}$ since we have $n$
choices for selecting the target line and for each of the remaining $n-1$ lines
three choices of either having it uncontrolled, positive controlled, or negative
controlled.  The number of self-inverse functions on $n$ variables that are
transpositions is $\big({{2^n}\atop 2}\big)$.  Finally, the number of
self-inverse functions on $n$ variables that are palindromic is
$$ \sum_{k=0}^{n-1} {1\over{2^k!}} \prod_{i=0}^{2^k-1} \left(\matrix{2^n-2i \cr 2}\right).\eqno(4) $$



\topinsert \centerline{Table 1: Number of self-inverse functions}

\ninepoint
\noindent\hfil
\vbox{
\offinterlineskip
\hrule
\tabskip=0em plus2em minus.5em
\halign{\strut\enspace#\hfil & \quad\hfil# & \quad\hfil# & \quad\hfil# & \quad\hfil# & \quad\hfil# & \quad\hfil# \cr
                   & $n=1$ & $n=2$ & $n=3$ & $n=4$ & $n=5$ & $n=6$ \cr
  reversible       & 2 & 24 & 40,320 & 20,922,789,888,000 & $>2\cdot 10^{35}$ & $>10^{89}$ \cr
  self-inverse     & 2 & 10 &    764 &         46,206,736 & $>2\cdot 10^{19}$ & $>10^{47}$ \cr
  palindromic      & 1 &  9 &    343 &          3,383,955 &        $>10^{17}$ & $>10^{44}$ \cr
  transpositions   & 1 &  6 &     28 &                120 &              496 &     2,016 \cr
  simple           & 1 &  6 &     27 &                108 &              405 &     1,458 \cr
  fully-controlled & 1 &  4 &     12 &                 32 &               80 &       192 \cr
}
\hrule}
\endinsert

All functions that have been discussed in this section are counted in Table 1
for one to six variables.  Note that the subset relation
$$
  \hbox{reversible} \supset \hbox{self-inverse} \supset \hbox{palindromic} \supset \hbox{transpositions} \eqno(5)
$$
holds.  We have `$\hbox{simple}\not\subset\hbox{transpositions}$' but
`$\hbox{simple}\subset\hbox{palindromic}$'.



\medskip \noindent{\bf 1.1. Mixed-polarity fully-controlled Toffoli gates.}\enspace We define a mixed-polarity fully-controlled Toffoli gate as a Toffoli gate in which all possible ($n - 1$) controls are used. For each $n$ target lines there are $2^{n-1}$ possible ways to set the two states of the $n-1$ controls and, thus, there are $n \cdot 2^{n-1}$ possible gates.

\smallskip \noindent{\bf Lemma 2.} A self-inverse function $f$ is both {\it simple\/} and {\it transposition\/} iff $f$ can be realized using a single mixed-polarity fully-controlled ($2^n-1$ controls) Toffoli gate.

\smallskip\noindent {\it Proof.} All mixed-polarity fully-controlled Toffoli gates have a Hamming distance of one and it thus implementable using only one transposition. \qquad\slug

Given that this is all fully-controlled Toffoli gates this is set is universal or, in other words, a generating set for all other rest of reversible functions. We know how the circuit look, but how can we characterise these gates as transpositions?

\smallskip \noindent{\bf Definition 1.} Given a transposition $(a\,b)$ we define the {\it length} of the transposition to be absolute difference of $a$ and $b$:
$$
length (a\,b) = |a - b|
$$

Now we divide them into $n$ classes (denoted $T_i$) each containing $2^{n-1}$ transpositions, where all transpositions in a class compute a gate with the target on the $i$'th line, $i \in {0,...,n-1}$.

\smallskip \noindent{\bf Definition 2.} The class $T_i$, $i \in \{0,...,n-1\}$, is equal to the set of transpositions $(a\,b)$ where 
$$
\{(a\,b) \,|\, a = a_1 + a_2 \cdot 2^i, a_1 \in \{0,..,2^{i-1}-1\}, a_2 \in \{0,..,2^{n-i}-1\}, length (a\,b) = 2^{i-1} \}
$$

In other words a transposition $(a\,b)$ is in $T_i$ iff the bit-wise exclusive-or product of the binary representations $a$ and $b$ have a 1 one the $i$'th position and 0 on the rest (or equal to $2^i$). The other notes shows this represented as a graph of permutations.

Note that all transposition of a class are disjoint and, thus, all numbers from 0 to $n-1$ is represented exactly ones in a transposition.

\smallskip\noindent {\bf Example 1.} See the other notes for examples for $n$ equal to 2 and 3. 

\smallskip \noindent{\bf Lemma 2.2.} A $k$-controlled mixed-polarity Toffoli gate with target on line $i$ is represented by $2^{n-k-1}$ transpositions from class $T_i$.

\smallskip\noindent {\it Proof.} Follows from [SoekenThomsen (2013)]. \qquad\slug

\smallskip \noindent{\bf Lemma 2.3.} Any transposition $(a\,b)$ can be represented by a palindromic circuit of $2^h-1$ transpositions from $h$ different classes $T_i$, where $h$ is the Hamming distance of the binary representation between $a = a_0...a_n$ and $b = b_0...b_n$, and $i \in \{ j \,|\, a_j \oplus b_j = 1, 0\leq j \leq n\}$ ($i$ is given by the bit positions in which $a$ and $b$ differs.

\smallskip\noindent {\it Proof.} Given a transposition $(a\,b)$, then it is all shortest paths in the graph notation between nodes $a$ to $b$ \qquad\slug




\medskip \noindent{\bf 2. Palindromic circuits.}\enspace A circuit
$C=g_1g_2\dots g_k$, which consists of mixed-polarity multiple-controlled
Toffoli gates $g_i$, is called {\it palindromic\/} if $ g_i = g_{k+1-i}$ for all
$i \in \{1,\dots,\lfloor{k\over 2}\rfloor\}$.  The circuit is called {\it
even\/} if $k$ is even and {\it odd\/} otherwise.

\smallskip \noindent{\bf Lemma 3.} \sl A palindromic circuit is even if and only
if it realizes the identity function. \rm

\smallskip\noindent {\it Proof.} Let $C=g_1g_2\dots g_{2k}$ be an even
palindromic circuit.  Then $g_{k}=g_{2k+1-k}=g_{k+1}$ and $C$ is functionally
equivalent to the circuit $g_1g_2\dots g_{k-1}g_{k+2} \dots g_{2k}$.  Again, the
two ``middle gates'' are equal and therefore we can continue this procedure
until no gates remain and we have that $C$ represents the identity function.

Now let $C=g_1g_2\dots g_kg_{k+1}g_{k+2}\dots g_{2k+1}$ be an odd palindromic
circuit and let $h$ be the function that is represented by the first $k$ gates
$g_1g_2\dots g_k$.  Since $C$ is palindromic the last $k$ gates $g_{k+2}\dots
g_{2k+1}$ represents the function $h^{-1}$.  Since $g_{k+1}$ is a Toffoli gate
there exists some $x$ such that $g_{k+1}(x)=y\neq x$.  Propagating $x$ to the
left yields $h^{-1}(x)$ at the inputs of $C$ and propagating $y$ to the right
yiels $h^{-1}(y)$ at the outputs of $C$.  Since $x\neq y$ we also have
$h^{-1}(x)\neq h^{-1}(y)$ and hence $C$ does not represent the identity
function.\qquad\slug

\noindent{\bf Lemma 4.} \sl A self-inverse function is a palindromic function if
and only if it can be realized by an odd palindromic circuit. \rm

\smallskip\noindent {\it Proof.} Given an odd palindromic circuit.  Then it
realizes a self-inverse function which permutation representation is $f\circ
g\circ f^{-1}$ with $g$ being a permutation representing a single Toffoli
gate. We know that ${\rm cyc}(g)=2^k$ for some nonnegative integer $k$.
According to Theorem 9.131 in [N.\ A.\ Loehr, {\sl Bijective Combinatorics\/}
(2011)] we have ${\rm cyc}(f\circ g\circ f^{-1}) = {\rm cyc}(g)$ and therefore
$f\circ g\circ f^{-1}$ is palindromic.

\dots \qquad\slug

% \smallskip \noindent{\bf Open question 1.} \sl Assuming that there exists a
% palindromic circuit for each self-inverse function, the question is whether
% there also exists a palindromic circuit in a V-shape for each self-inverse
% function. \rm

% \medskip\noindent{\bf 2. Connection to permutations.}\enspace

% Working with involutions, we can find an equivalent conjecture to Conjecture~1
% based on the notation of permutations.  First, we prove a further lemma.

% \smallskip\noindent{\bf Lemma 4.} \sl A permutation $\pi$ is an involution if
% and only if it can be written as $\pi = \pi_1\circ\pi_2\circ\pi_1^{-1}$ where
% $\pi_1\neq \pi_{\rm id}$ and $\pi_2\neq \pi$ is an involution. \rm

% \smallskip\noindent {\it Proof.}  Let $\pi = \pi_1\circ\pi_2\circ\pi_1^{-1}$ and
% let $\pi_1(x)=y$, $\pi_2(y)=z$, and $\pi_1^{-1}(z)=w$.  Since $\pi_2$ is an
% involution we have $\pi(x)=w$ and $\pi(w)=x$ and therefore $\pi$ is also an
% involution.

% Now assume that $\pi$ is an involution.  From the previous proof direction we
% obtain another involution $\pi_2\neq \pi$ by taking any permutation $\pi_1\neq
% \pi_{\rm id}$ by calculating $\pi_2=\pi_1\circ\pi\circ \pi_1^{-1}$.  Multiplying
% the left side of the equation with $\pi_1^{-1}$ and the right side with $\pi_1$
% we obtain $\pi=\pi_1^{-1}\circ\pi_2\circ\pi_1$, concluding the
% proof. \qquad\slug

% \smallskip\noindent {\bf Conjecture 2.} \sl Every involution $\pi$ can be
% decomposed into $\pi = \pi_1\circ \pi_g\circ \pi_1^{-1}$ where $\pi_g$ is a
% permutation that can be realized using a single Toffoli gate. \rm

% \smallskip\noindent Conjecture 2 is equivalent to Conjecture 1.

\medskip\noindent{\bf Transpositions.}\enspace Any transposition $(a,b)$ can be
realized with $2h-1$ gates where $h$ is the hamming distance between the binary
expansions of $a=a_1a_2\dots a_n$ and $b=b_1b_2\dots b_n$.  More precisely, the
circuit consists of $2(h-1)$ CNOT gates and one fully controlled Toffoli gate.
In the worst case we have $h=n$.  The circuit to represent a transposition has
the form $ABA$ and is constructed as follows: Let $i_1,\dots,i_h$ be the indexes
in which the binary expansions of $a$ and $b$ differ.  Part $A$ consists of CNOT
gates ${\rm T}(\{x_h^{a_h}\}, x_j)$ for $1\le j< h$ and part $B$ consists of the
single Toffoli gate ${\rm T}(\{x_j^{b_j}\mid j\neq h\}, x_h)$.  Note that all
gates in part $A$ can be positioned in any order and hence $A=A^{-1}$ for each
chosen order.  Given $a=001$ and $b=110$ we can choose $h \in \{1,2,3\}$ and get
the following circuits:

$$
\tikzpicture[baseline=(c)]
\draw[line width=0.30] (0.50,1.25) -- (3.00,1.25);
\draw[line width=0.30] (0.50,0.75) -- (3.00,0.75);
\draw[line width=0.30] (0.50,0.25) -- (3.00,0.25);
\draw[line width=0.30] (0.75,0.25) -- (0.75,1.25);
\draw[fill] (0.75,0.25) circle (0.10);
\draw[line width=0.3] (0.75,1.25) circle (0.2) (0.75,1.05) -- (0.75,1.45);
\draw[line width=0.30] (1.25,0.25) -- (1.25,0.75);
\draw[fill] (1.25,0.25) circle (0.10);
\draw[line width=0.3] (1.25,0.75) circle (0.2) (1.25,0.55) -- (1.25,0.95);
\draw[line width=0.30] (1.75,0.25) -- (1.75,1.25);
\draw[fill] (1.75,1.25) circle (0.10);
\draw[fill] (1.75,0.75) circle (0.10);
\draw[line width=0.3] (1.75,0.25) circle (0.2) (1.75,0.05) -- (1.75,0.45);
\draw[line width=0.30] (2.25,0.25) -- (2.25,0.75);
\draw[fill] (2.25,0.25) circle (0.10);
\draw[line width=0.3] (2.25,0.75) circle (0.2) (2.25,0.55) -- (2.25,0.95);
\draw[line width=0.30] (2.75,0.25) -- (2.75,1.25);
\draw[fill] (2.75,0.25) circle (0.10);
\draw[line width=0.3] (2.75,1.25) circle (0.2) (2.75,1.05) -- (2.75,1.45);
\coordinate (c) at ($(current bounding box.north west)!.5!(current bounding box.south east)$);
\endtikzpicture
\qquad\qquad
\tikzpicture[baseline=(c)]
\draw[line width=0.30] (0.50,1.25) -- (3.00,1.25);
\draw[line width=0.30] (0.50,0.75) -- (3.00,0.75);
\draw[line width=0.30] (0.50,0.25) -- (3.00,0.25);
\draw[line width=0.30] (0.75,0.75) -- (0.75,1.25);
\draw[fill=white] (0.75,0.75) circle (0.10);
\draw[line width=0.3] (0.75,1.25) circle (0.2) (0.75,1.05) -- (0.75,1.45);
\draw[line width=0.30] (1.25,0.25) -- (1.25,0.75);
\draw[fill=white] (1.25,0.75) circle (0.10);
\draw[line width=0.3] (1.25,0.25) circle (0.2) (1.25,0.05) -- (1.25,0.45);
\draw[line width=0.30] (1.75,0.25) -- (1.75,1.25);
\draw[fill] (1.75,1.25) circle (0.10);
\draw[fill=white] (1.75,0.25) circle (0.10);
\draw[line width=0.3] (1.75,0.75) circle (0.2) (1.75,0.55) -- (1.75,0.95);
\draw[line width=0.30] (2.25,0.25) -- (2.25,0.75);
\draw[fill=white] (2.25,0.75) circle (0.10);
\draw[line width=0.3] (2.25,0.25) circle (0.2) (2.25,0.05) -- (2.25,0.45);
\draw[line width=0.30] (2.75,0.75) -- (2.75,1.25);
\draw[fill=white] (2.75,0.75) circle (0.10);
\draw[line width=0.3] (2.75,1.25) circle (0.2) (2.75,1.05) -- (2.75,1.45);
\coordinate (c) at ($(current bounding box.north west)!.5!(current bounding box.south east)$);
\endtikzpicture
\qquad\qquad
\tikzpicture[baseline=(c)]
\draw[line width=0.30] (0.50,1.25) -- (3.00,1.25);
\draw[line width=0.30] (0.50,0.75) -- (3.00,0.75);
\draw[line width=0.30] (0.50,0.25) -- (3.00,0.25);
\draw[line width=0.30] (0.75,0.75) -- (0.75,1.25);
\draw[fill=white] (0.75,1.25) circle (0.10);
\draw[line width=0.3] (0.75,0.75) circle (0.2) (0.75,0.55) -- (0.75,0.95);
\draw[line width=0.30] (1.25,0.25) -- (1.25,1.25);
\draw[fill=white] (1.25,1.25) circle (0.10);
\draw[line width=0.3] (1.25,0.25) circle (0.2) (1.25,0.05) -- (1.25,0.45);
\draw[line width=0.30] (1.75,0.25) -- (1.75,1.25);
\draw[fill] (1.75,0.75) circle (0.10);
\draw[fill=white] (1.75,0.25) circle (0.10);
\draw[line width=0.3] (1.75,1.25) circle (0.2) (1.75,1.05) -- (1.75,1.45);
\draw[line width=0.30] (2.25,0.25) -- (2.25,1.25);
\draw[fill=white] (2.25,1.25) circle (0.10);
\draw[line width=0.3] (2.25,0.25) circle (0.2) (2.25,0.05) -- (2.25,0.45);
\draw[line width=0.30] (2.75,0.75) -- (2.75,1.25);
\draw[fill=white] (2.75,1.25) circle (0.10);
\draw[line width=0.3] (2.75,0.75) circle (0.2) (2.75,0.55) -- (2.75,0.95);
\coordinate (c) at ($(current bounding box.north west)!.5!(current bounding box.south east)$);
\endtikzpicture
\eqno(6)
$$

\noindent Since $(a,b)=(b,a)$ we also get the following three circuits:
$$
\tikzpicture[baseline=(c)]
\draw[line width=0.30] (0.50,1.25) -- (3.00,1.25);
\draw[line width=0.30] (0.50,0.75) -- (3.00,0.75);
\draw[line width=0.30] (0.50,0.25) -- (3.00,0.25);
\draw[line width=0.30] (0.75,0.25) -- (0.75,1.25);
\draw[fill=white] (0.75,0.25) circle (0.10);
\draw[line width=0.3] (0.75,1.25) circle (0.2) (0.75,1.05) -- (0.75,1.45);
\draw[line width=0.30] (1.25,0.25) -- (1.25,0.75);
\draw[fill=white] (1.25,0.25) circle (0.10);
\draw[line width=0.3] (1.25,0.75) circle (0.2) (1.25,0.55) -- (1.25,0.95);
\draw[line width=0.30] (1.75,0.25) -- (1.75,1.25);
\draw[fill=white] (1.75,1.25) circle (0.10);
\draw[fill=white] (1.75,0.75) circle (0.10);
\draw[line width=0.3] (1.75,0.25) circle (0.2) (1.75,0.05) -- (1.75,0.45);
\draw[line width=0.30] (2.25,0.25) -- (2.25,0.75);
\draw[fill=white] (2.25,0.25) circle (0.10);
\draw[line width=0.3] (2.25,0.75) circle (0.2) (2.25,0.55) -- (2.25,0.95);
\draw[line width=0.30] (2.75,0.25) -- (2.75,1.25);
\draw[fill=white] (2.75,0.25) circle (0.10);
\draw[line width=0.3] (2.75,1.25) circle (0.2) (2.75,1.05) -- (2.75,1.45);
\coordinate (c) at ($(current bounding box.north west)!.5!(current bounding box.south east)$);
\endtikzpicture
\qquad\qquad
\tikzpicture[baseline=(c)]
\draw[line width=0.30] (0.50,1.25) -- (3.00,1.25);
\draw[line width=0.30] (0.50,0.75) -- (3.00,0.75);
\draw[line width=0.30] (0.50,0.25) -- (3.00,0.25);
\draw[line width=0.30] (0.75,0.75) -- (0.75,1.25);
\draw[fill] (0.75,0.75) circle (0.10);
\draw[line width=0.3] (0.75,1.25) circle (0.2) (0.75,1.05) -- (0.75,1.45);
\draw[line width=0.30] (1.25,0.25) -- (1.25,0.75);
\draw[fill] (1.25,0.75) circle (0.10);
\draw[line width=0.3] (1.25,0.25) circle (0.2) (1.25,0.05) -- (1.25,0.45);
\draw[line width=0.30] (1.75,0.25) -- (1.75,1.25);
\draw[fill=white] (1.75,1.25) circle (0.10);
\draw[fill] (1.75,0.25) circle (0.10);
\draw[line width=0.3] (1.75,0.75) circle (0.2) (1.75,0.55) -- (1.75,0.95);
\draw[line width=0.30] (2.25,0.25) -- (2.25,0.75);
\draw[fill] (2.25,0.75) circle (0.10);
\draw[line width=0.3] (2.25,0.25) circle (0.2) (2.25,0.05) -- (2.25,0.45);
\draw[line width=0.30] (2.75,0.75) -- (2.75,1.25);
\draw[fill] (2.75,0.75) circle (0.10);
\draw[line width=0.3] (2.75,1.25) circle (0.2) (2.75,1.05) -- (2.75,1.45);
\coordinate (c) at ($(current bounding box.north west)!.5!(current bounding box.south east)$);
\endtikzpicture
\qquad\qquad
\tikzpicture[baseline=(c)]
\draw[line width=0.30] (0.50,1.25) -- (3.00,1.25);
\draw[line width=0.30] (0.50,0.75) -- (3.00,0.75);
\draw[line width=0.30] (0.50,0.25) -- (3.00,0.25);
\draw[line width=0.30] (0.75,0.75) -- (0.75,1.25);
\draw[fill] (0.75,1.25) circle (0.10);
\draw[line width=0.3] (0.75,0.75) circle (0.2) (0.75,0.55) -- (0.75,0.95);
\draw[line width=0.30] (1.25,0.25) -- (1.25,1.25);
\draw[fill] (1.25,1.25) circle (0.10);
\draw[line width=0.3] (1.25,0.25) circle (0.2) (1.25,0.05) -- (1.25,0.45);
\draw[line width=0.30] (1.75,0.25) -- (1.75,1.25);
\draw[fill=white] (1.75,0.75) circle (0.10);
\draw[fill] (1.75,0.25) circle (0.10);
\draw[line width=0.3] (1.75,1.25) circle (0.2) (1.75,1.05) -- (1.75,1.45);
\draw[line width=0.30] (2.25,0.25) -- (2.25,1.25);
\draw[fill] (2.25,1.25) circle (0.10);
\draw[line width=0.3] (2.25,0.25) circle (0.2) (2.25,0.05) -- (2.25,0.45);
\draw[line width=0.30] (2.75,0.75) -- (2.75,1.25);
\draw[fill] (2.75,1.25) circle (0.10);
\draw[line width=0.3] (2.75,0.75) circle (0.2) (2.75,0.55) -- (2.75,0.95);
\coordinate (c) at ($(current bounding box.north west)!.5!(current bounding box.south east)$);
\endtikzpicture
\eqno(7)
$$

\smallskip\noindent{\bf Lemma 5.} \sl One requires at least $2h-1$ gates to
represent a transpositon $(a,b)$ when the hamming distance between the binary
expansion of $a$ and $b$ is $h$. \rm 

\smallskip\noindent {\it Proof.} We proof it by induction on $h$.  Since $(a,b)$
is a transposition, we have $a\neq b$ and hence $h>0$.  For the base case $h=1$,
obviously at least one gate is required since otherwise the circuit realizes the
identity.  We assume that the lemma holds for $h$ and now consider the case
$h+1$.  In this case the hamming distance between $a$ and $b$ is $h+1$. \dots

\medskip\noindent {\bf 3. Synthesis of self-inverse functions.}\enspace \dots

\medskip\noindent {\bf 4. Lower and upper bounds.}\enspace \dots

\bye
