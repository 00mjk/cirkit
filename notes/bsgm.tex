\let\safetikz\shipout
\input pgf
\input tikz
\usetikzlibrary{calc,matrix}
\tikzset{>=stealth}
\let\shipout\safetikz

\magnification\magstephalf
\parskip3pt
\baselineskip14pt

\catcode`@=11
\def\oldstyle{\fam\@ne\teni}

\def\bbbb{{\rm I\!B}}

\def\slug{\hbox{\kern1.5pt\vrule width2.5pt height6pt depth1.5pt\kern1.5pt}}
\def\slugonright{\vrule width0pt\nobreak\hfill\slug}

% Equations
\def\eqex{1}
\def\eqexadjmat{2}
\def\eqmap{3}
\def\eqexva{4}
\def\eqfuncsem{5}
\def\eqonehota{6}
\def\eqonehotb{7}
\def\eqinjsem{8}
\def\eqonehotc{9}
\def\eqhvar{10}
\def\eqhmatrix{11}
\def\eqhjl{12}
\def\eqttrue{13}
\def\eqtfalse{14}
\def\eqhunitclause{15}
\def\eqhclause{16}

\centerline{\bf Reverse engineering using simulation graphs and bipartite subgraph isomorphism}
\centerline{Mathias Soeken, University of Bremen}
\centerline{Bremen, September 4, 2014}
\bigskip\bigskip

\noindent {\bf 1. Bipartite subgraph isomorphism.} A {\it unidirected\/} bigraph
is a digraph $G=(V,A)$ for which $V$ is a partition of two disjoint sets of
sources $V'$ and sinks $V''$ and arcs $A\subseteq V'\times V''$, i.e.~the
initial and final vertex of each arc lies in $V'$ and $V''$, respectively.  The
adjacency matrix of such a unidirected bigraph always has the form
$M=\big({O\atop O}{M'\atop O}\big)$ where each $O$ is a matrix containing only
zeros and the rows and columns of $M$ are arranged in a way that first vertices
are taken from $V'$ and then from $V''$.  Consequently, we can drop the $O$
matrices and end up with the {\it compact\/} adjacency matrix~$M'$.  Given two
unidirected bigraphs called {\it target\/} $G=(V,A)$ and {\it pattern\/}
$P=(W,B)$, the {\it bipartite subgraph isomorphism\/} problem asks whether there
exists a subgraph $G'$ of $G$ such that~$G'$ is isomorphic to $P$, denoted
$G'\cong P$.  In the remainder, we will use the graphs
$$
  G=\quad
  \tikzpicture[baseline=(c.base),every node/.style={circle,draw,minimum size=.5cm,inner sep=1pt}]
    \node (v1) at (0,0) {$1$};
    \node (v2) at (0,-.75) {$2$};
    \node (v3) at (0,-1.5) {$3$};
    \node (v4) at (0,-2.25) {$4$};
    \node (w1) at ([xshift=1.5cm] $(v1)!.5!(v2)$) {$\bar 1$};
    \node (w2) at ([xshift=1.5cm] $(v2)!.5!(v3)$) {$\bar 2$};
    \node (w3) at ([xshift=1.5cm] $(v3)!.5!(v4)$) {$\bar 3$};

    \draw[->] (v1) to[bend left=10] (w1);
    \draw[->] (v1) -- (w2);
    \draw[->] (v2) -- (w1);
    \draw[->] (v2) -- (w3);
    \draw[->] (v3) -- (w2);
    \draw[->] (v4) to[bend left=10] (w1);
    \draw[->] (v4) to[bend right=10] (w2);
    \coordinate (c) at (current bounding box.west);
  \endtikzpicture
  \qquad\hbox{and}\qquad
  P=\quad
  \tikzpicture[baseline=(c.base),every node/.style={circle,draw,minimum size=.5cm,inner sep=1pt}]
    \node (v1) at (0,0) {$1$};
    \node (v2) at (0,-.75) {$2$};
    \node (v3) at (0,-1.5) {$3$};
    \node (w1) at ([xshift=1.5cm] $(v1)!.5!(v2)$) {$\bar 1$};
    \node (w2) at ([xshift=1.5cm] $(v2)!.5!(v3)$) {$\bar 2$};

    \draw[->] (v1) to[bend left=10] (w1);
    \draw[->] (v1) to[bend right=5] (w2);
    \draw[->] (v2) -- (w1);
    \draw[->] (v3) -- (w1);
    \draw[->] (v3) to[bend right=10] (w2);

    \coordinate (c) at (current bounding box.west);
  \endtikzpicture
  \eqno(\eqex)
$$
as target and pattern in a running example.  Their compact adjacency matrices
are
$$
  M_G = \pmatrix{1 & 1 & 0 \cr 1 & 0 & 1 \cr 0 & 1 & 0 \cr 1 & 1 & 0}
  \qquad\hbox{and}\qquad
  M_P = \pmatrix{1 & 1 \cr 1 & 0 \cr 1 & 1}.
  \eqno(\eqexadjmat)
$$
It can readily be verified, that there exists a subgraph isomorphism between
these two graphs.

\noindent {\bf SAT formulation.} We will now create a Boolean function $f$ that
is satisfiable if and only if $P$ is isomorphic to a subgraph of $G$.  In case
there exists such a subgraph it can be obtained from a satisfying assignment to
$f$.  We assume that $G$ has $n+q$ vertices; $n$ sources and $q$ sinks.
Analogously, $P$ has $m+r$ vertices.  Obviously, we have $m\le n$ and $r\le q$.

First, we are looking to the bipartite subgraph isomorphism problem from a
different angle.  Let $g^{(1)},\dots,g^{(n)}$ and $p^{(1)},\dots,p^{(m)}$ be the
row vectors of the compact adjacency matrices of the target $G$ and pattern $P$.
Their bitwidths are $q$ and $r$, respectively.  The existence of a bipartite
subgraph isomorphism is equivalent to the existence of two injective functions
$v:[m]\to[n]$ and $a:[r]\to[q]$ such that
$$ p^{(j)}_l=g^{(v(j))}_{a(l)} \eqno(\eqmap) $$
for all $1\le j\le m$ and $1\le l\le r$.  (The notation $[n]$ is a shorthand for
the set $\{1,\dots,n\}$.)  The function $v$ maps each source in $P$ to a
distinct source in $G$, while the function $a$ maps each sink in $P$ to a
distinct sink in $G$.  For the example in (\eqex) we have
$$
  v(1)=4,\qquad v(2)=3,\qquad v(3)=1,\qquad
  a(\bar 1)=\bar 2,\qquad a(\bar 2)=\bar 1.
  \eqno(\eqexva)
$$
(The barred notation for values in $a$ has intentionally been used to improve
comprehensibility.)

This equivalent formulation makes it easy for us to describe the satisfiability
function $f$.  The function $v$ is described by $mn$ variables $v_{j,i}$ such
that $v(j)=i$ if and only if $v_{j,i}=1$ for all $1\le j\le m$ and $1\le i\le
n$; in other words we have $v_{j,v(j)}=1$.  To ensure function semantics,
i.e.~each $v(j)$ is assigned exactly one value, clauses for the Boolean
functions
$$ S_1(v_{j,1}, \dots, v_{j,n}) \qquad\hbox{for $1\le j\le m$} \eqno(\eqfuncsem) $$
are added.  ($S_1(x_1,\dots,x_n)$ describes the symmetric Boolean function that
is true if and only if exactly one of the $n$ variables is true; see
Eq.~7.1.1-({\oldstyle90}) in TAOCP.)  The encoding is sometimes referred to as
{\it one hot encoding\/} and can be enforced by the clauses
$$ v_{j,1}\lor\cdots\lor v_{j,n} \qquad \hbox{for $1\le j\le m$} \eqno(\eqonehota) $$
to ensure that at least one bit is set to 1 together with the exclusion clauses
$$ \bar v_{j,i}\lor \bar v_{j,i'} \qquad \hbox{for $1\le j\le m$ and $1\le i<i'\le n$} \eqno(\eqonehotb) $$
that ensure that at most one bit is set to 1.
The injectivity property can be enforced by adding clauses for the functions
$$ S_{\le1}(v_{1,i},\dots,v_{m,i}) \qquad\hbox{for $1\le i\le n$}. \eqno(\eqinjsem) $$
This ensures that for each $i\in\{1,\dots,n\}$ there is at most one $j$ such
that $v(j)=i$ and can be encoded using only the exclusion clauses
$$ \bar v_{j,i}\lor \bar v_{j',i} \qquad \hbox{for $1\le j<j'\le m$ and $1\le i\le n$}. \eqno(\eqonehotc) $$
Analogously, $rq$ variables $a_{l,k}$ describe function $a$ for $1\le l\le r$
and $1\le k\le q$ and similar clauses to (\eqonehota), (\eqonehotb), and
(\eqonehotc) ensure the correct semantics.

We are now getting to the core of (\eqmap).  We will eventually add clauses for
all function values of $v$ and $a$, i.e.~for all combinations of
$j\in\{1,\dots,m\}$ and $l\in\{1,\dots,r\}$ , but let us first take a look on
how to represent the right hand side in (\eqmap) using the variables we have
added so far.  For this purpose $nmqr$ variables $h_{i,j,k,l}$ are added for
which
$$h_{i,j,k,l} = g^{(i)}_k\land v_{j,i}\land a_{l,k} \eqno(\eqhvar)$$
should hold.  These will be the most complicated variables in our encoding and
it requires some further discussion.  When we fix $j$ and $l$ and we range over
all $i\in\{1,\dots,n\}$ and $k\in\{1,\dots,q\}$, the values for $h_{i,j,k,l}$
are the components of the result of the bitwise AND of the bitvectors $\hat g$,
$\hat v$, and $\hat a$:
$$
\tikzpicture[baseline=(c.base)]
  \matrix[matrix of math nodes] (m) {
  g^{(1)}_1 & g^{(1)}_2 & \cdots & g^{(1)}_q & g^{(2)}_1 & g^{(2)}_2 & \cdots & g^{(2)}_q & \cdots & g^{(n)}_1 & g^{(n)}_2 & \cdots & g^{(n)}_q \\
  \land & \land & & \land & \land & \land & & \land & & \land & \land & & \land \\
  v_{j,1} & v_{j,1} & \cdots & v_{j,1} & v_{j,2} & v_{j,2} & \dots & v_{j,2} & \cdots & v_{j,n} & v_{j,n} & \dots & v_{j,n} \\
  \land & \land & & \land & \land & \land & & \land & & \land & \land & & \land \\
  a_{l,1} & a_{l,2} & \cdots & a_{l,q} & a_{l,1} & a_{l,2} & \cdots & a_{l,q} & \cdots & a_{l,1} & a_{l,2} & \cdots & a_{l,q} \\
  };

  \node[left] at (m-1-1.east) {$\hat g\;=\;\phantom{g^{(1)}_1}$};
  \node[left] at (m-3-1.east) {$\hat v\;=\;\phantom{v_{j,1}}$};
  \node[left] at (m-5-1.east) {$\hat a\;=\;\phantom{a_{l,1}}$};

  \node[rotate=270,left] at (m-1-1.north) {$h_{1,j,1,l}=$};
  \node[rotate=270,left] at (m-1-2.north) {$h_{1,j,2,l}=$};
  \node[rotate=270,left] at (m-1-4.north) {$h_{1,j,q,l}=$};
  \node[rotate=270,left] at (m-1-5.north) {$h_{2,j,1,l}=$};
  \node[rotate=270,left] at (m-1-6.north) {$h_{2,j,2,l}=$};
  \node[rotate=270,left] at (m-1-8.north) {$h_{2,j,q,l}=$};
  \node[rotate=270,left] at (m-1-10.north) {$h_{n,j,1,l}=$};
  \node[rotate=270,left] at (m-1-11.north) {$h_{n,j,2,l}=$};
  \node[rotate=270,left] at (m-1-13.north) {$h_{n,j,q,l}=$};

  \coordinate (c) at (current bounding box.west);
\endtikzpicture
\eqno(\eqhmatrix)
$$
The bitvector $\hat g$ results from the concatenation of all row vectors from
the compact adjecency matrix of $G$.  The bitvector $\hat v$ is a mask of form
$0\dots 01\dots 10\dots 0$,
%more precisely $0^{(v(j)-1)p}1^p0^{(n-v(j))p}$,
where we have ones only for the entries $v_{j,v(j)}$.  Hence, $\hat v$ selects
the correct row vector $g^{(v(j))}$.  Finally, $\hat a = (a_{l,1}a_{l,2}\dots
a_{l,q})^n$ is a mask with $n$ copies of $a_{l,1}a_{l,2}\dots a_{l,q}$ which
only 1 is at position $a_{l,a(l)}$.  We can see that $n-1$ of these copies are
discarded by $\hat v$ and only the $v(j)$-th copy remains.  From this copy it
selects the bit $g^{(v(l))}_{a(l)}$.  We collect all the bits $h_{i,j,k,l}$ in
(\eqhmatrix) in the set
$$H_{j,l} = \{h_{i,j,k,l}\mid1\le i\le n, 1\le k\le q\}. \eqno(\eqhjl) $$
With the previous considerations it is easy to see that there is at most one
variable in $H_{j,l}$ that is true in a satisfying assignment.  Moreover, there
exists one true variable in $H_{j,l}$ if and only if $g^{(v(j))}_{a(l)}=1$.
Now, we are ready to describe the main clauses that we add for all
$j\in\{1,\dots,m\}$ and $l\in\{1,\dots,r\}$.  If $p^{(j)}_l=1$ we add one large clause
$$ \bigvee_{h\in H_{j,l}} h \eqno(\eqttrue) $$
with $nq$ literals to ensure that also $g^{(v(j))}_{a(l)}=1$, and if we have
$p^{(j)}_l=0$ we add the $nq$ unit clauses
$$ \bigwedge_{h\in H_{j,l}} \bar h \eqno(\eqtfalse) $$
to ensure that also $g^{(v(j))}_{a(l)}=0$.  There is one last thing to do, since
we have ignored to add clauses for $h_{i,j,k,l}$.  Since the values for
$g^{(i)}_k$ are constant in (\eqhvar) we make a case distinction on them.  If
$g^{(i)}_k=0$ we just add the unit clause
$$ \bar h_{i,j,k,l} \eqno(\eqhunitclause) $$
and if $g^{(i)}_k=1$, we add the three clauses
$$ (h_{i,j,k,l}\lor \bar v_{j,i} \lor \bar a_{l,k}) \land (\bar h_{i,j,k,l}\lor v_{j,i}) \land (\bar h_{i,j,k,l} \lor a_{l,k}). \eqno(\eqhclause) $$

\medskip\noindent{\bf Complexity of the formula.}  Now we want to look into the
size of the resulting formulas.  There are in total $mn+rq+nmqr\le O(|V|^4)$
variables $v_{j,i}$, $a_{l,k}$, and $h_{i,j,k,l}$.  The total number of clauses
is
$$
  \kappa(G,P)=
  \underbrace{m+r}_{(\eqonehota)}
  + \underbrace{\left(m\atop 2\right)+\left(r\atop 2\right)}_{(\eqonehotb)}
  + \underbrace{\left(n\atop 2\right)+\left(q\atop 2\right)}_{(\eqonehotc)}
  + \underbrace{|B|}_{(\eqttrue)}
  + \underbrace{(mr-|B|)\cdot nq}_{(\eqtfalse)}
  + \underbrace{(nq-|A|)}_{(\eqhunitclause)}
  + \underbrace{|A|\cdot3}_{(\eqhclause)},
$$
where for (\eqonehota), (\eqonehotb), and (\eqonehotc) also the clauses to
encode function $a$ are taken into account.  In order to derive the number of
clauses for (\eqttrue)--(\eqhclause) take into account that the compact
adjacency matrix for $G$ has $|A|$ ones and $nq-|A|$ zeros.  Similarly, the
compact adjacency matrix for $P$ has $|B|$ ones and $mr-|B|$ zeros.  Considering
that $|B|\le |A|$, $m\le n$, and $r\le q$, we have
$$
  \eqalign{
  \kappa(G,P) &\le
  n+q
  + 2\cdot\left(n\atop 2\right)+2\cdot\left(q\atop 2\right)
  + |A|\cdot 3 + (nq-|A|)
  + |A| + (nq-|A|)\cdot nq \cr
  &=
  n+q
  + 2\cdot\left(n\atop 2\right)+2\cdot\left(q\atop 2\right)
  + |A|\cdot 3 +  nq-|A|
  + |A| + (nq)^2-|A| nq \cr
  &=
  n+q
  + 2\cdot\left(n\atop 2\right)+2\cdot\left(q\atop 2\right)
  + |A|\cdot(3-nq) +  nq
  + (nq)^2.}
$$

% \smallskip\noindent{\bf Lemma 1.} \sl At most one variable in $H_{j,l}$ can be
% true in a satisfying assignment. \rm

% {\it Proof.} First, we assume that no negative unit clauses such as in (??) have
% been added.  According to (2) there is exactly one $\hat\imath$ such that
% $v_{j,\hat\imath}$ is true and therefore all $h_{i,j,k,l}$ for $i\neq\hat\imath$
% can be discarded and we get $H\supseteq \hat H=\{v_{j,\hat\imath}\land
% a_{l,k}\mid 1\le k\le p\}$.  With a similar argumentation there is also only one
% $\hat k$ such that $a_{l,k}$ is true.\slugonright

\medskip\noindent{\bf Further reading.}  Jacobo Tor\'an presents a SAT encoding
for graph isomorphism in [{\sl SAT\/ \bf 16} (2013), 52--66].

The subgraph isomorphism problem is known to be NP-complete for general graphs
[M.\ R.\ Garey and D.\ S.\ Johnson, {\sl Computers and Intractibility} (1979)].
Effecient implementations for special cases are discussed in [J.\ E.\ Hopcroft
and J.\ K.\ Wong, {\sl STOC\/ \bf 6} (1974), 172--184] and [E.\ M.\ Luks, {\sl
JCSS\/ \bf 25} (1982), 42--65].

[C.\ Solnon, {\sl AI\/ \bf 174} (2010), 850--864]  [S.\ Zampelli and C.\ Solnon, {\sl Constraints\/ \bf 15} (2010), 327--353]

\medskip\noindent{\bf 2. Simulation graphs.}  Let $y=f(x)$ be a multibit
function which is defined over $m$ functions $f_1(x_1,\dots,x_n), \dots,
f_m(x_1,\dots,x_n)$.  In other words, $y=f_1\dots f_m$ is a binary vector of
length $m$ and $x=x_1\dots x_n$ is a binary vector of length $n$.  Given such a
multibit function together with simulation vectors $x^{(1)},\dots, x^{(k)}$ a
{\it simulation graph\/} is a unidirected bigraph $G=(V,A)$ with source vertices
$v_1,\dots,v_k$ for each simulation vector and sink vertices
$v_{k+1}, \dots, v_{k+m}$ for each output of $f$.  It has an arc
$$ v_i \longrightarrow v_{k+j}, \quad \hbox{if and only if} \quad f_j(x^{(i)})=1 $$
for $1\le i\le k$ and $1\le j\le m$.

\centerline{****}

\noindent{\bf Consideration of inverters.}

\centerline{****}

\noindent{\bf Symmetry breaking.}

\centerline{****}

\noindent{\bf Further improvements with preprocessing.}

\centerline{****}

\noindent{\bf Further reading.}
[J.\ A.\ Roy, I.\ L.\ Markov, and V.\ Bertacco, {\sl IWLS\/ \bf 13} (2004)]


\bye
<