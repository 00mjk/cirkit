\magnification\magstephalf
\parskip3pt
\baselineskip14pt

\def\band{\mathop{\hbox{\&}}}
\def\ind{\phantom{a}}

% macros for verbatim scanning
% the max number of \tt chars/line is 66 (10pt), 73 (9pt), 81 (8pt), 93 (7pt)
% minus 4 or 5 characters for indentation at the left
\chardef\other=12
\def\ttverbatim{\catcode`\\=\other
  \catcode`\{=\other
  \catcode`\}=\other
  \catcode`\$=\other
  \catcode`\&=\other
  \catcode`\#=\other
  \catcode`\%=\other
  \catcode`\~=\other
  \catcode`\_=\other
  \catcode`\^=\other
  \obeyspaces \obeylines \tt}
\def\begintt{$$\ttverbatim \catcode`\|=0 \ttfinish}
{\catcode`\|=0 \catcode`|\=\other % | is temporary escape character
  |obeylines   % end of line is active
  |gdef|intt#1^^M{|noalign{#1}}%
  |gdef|ttfinish#1^^M#2\endtt{#1|let^^M=|cr %
    |halign{|hskip|parindent##|hfil|cr#2}$$}}

\centerline{\bf An interesting bitwise operation}
\centerline{Mathias Soeken, University of Bremen}
\centerline{Bremen, August 8, 2014}
\bigskip\bigskip

Let `$\circ$' be an operation on two bitstrings of equal lengths that indicates
common ones projected to the positions where the first argument has ones.  Let
us illustrate this operation first by means of some examples (For more clarity,
the matched ones in the first argument are underlined):

$$ 010\underline1\,\underline1010 \circ 0001\,1000 = 0110 \qquad
   1\underline10\underline1\,1\underline100 \circ 0101\,0100 = 01101 $$

Please note that the operation is not commutative and that the second argument
is always contained in the first one, i.e.~$a\circ b$ is only well-defined if
$a\band b=b$.  We are seeking for an efficient implementation of~`$\circ$'.

A trivial implementation that iterates over all  1's of the first operand can be
given as follows.
\begintt
typedef boost::dynamic_bitset<> pattern_t;

pattern_t f( const pattern_t& a, const pattern_t& b )
{
|ind pattern_t p( a.count() );
|ind unsigned bpos = 0u;
|ind pattern_t::size_type pos = a.find_first();
|ind while ( pos != pattern_t::npos )
|ind {
|ind |ind p[bpos++] = b[pos];
|ind |ind pos = a.find_next( pos );
|ind }
|ind return p;
}
\endtt

\bye
